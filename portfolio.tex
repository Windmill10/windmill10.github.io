\documentclass[11pt,a4paper]{article}

% Document setup and packages
\usepackage[margin=0.75in]{geometry}
\usepackage{hyperref}
\usepackage{fontawesome5}
\usepackage{titlesec}
\usepackage{enumitem}
\usepackage{xcolor}
\usepackage{graphicx}
\usepackage{CJKutf8}
\usepackage{array}
\usepackage{tabularx}

% Color definitions
\definecolor{primary}{RGB}{79, 134, 247}
\definecolor{light}{RGB}{110, 165, 255}
\definecolor{dark}{RGB}{30, 41, 59}
\definecolor{text}{RGB}{51, 65, 85}

% Hyperlink setup
\hypersetup{
    colorlinks=true,
    urlcolor=primary,
    linkcolor=primary
}

% Title formatting
\titleformat{\section}
  {\Large\bfseries\color{primary}}
  {}{0em}
  {}
  [\titlerule]
  
\titleformat{\subsection}
  {\bfseries\color{dark}}
  {}{0em}
  {}

% Custom command for project title
\newcommand{\project}[1]{{\large\bfseries #1}}

% Custom command for skills
\newcommand{\skill}[2]{
  \textbf{#1} & #2 \\
}

% Begin document
\begin{document}
\begin{CJK}{UTF8}{bsmi}

% Header with name and contact information
\begin{center}
    {\Huge\color{primary}\textbf{李聿宸} (Yu Chen Lee)}\\[0.3cm]
    {\large Computer Science Student at National Tsing Hua University}\\[0.3cm]
    
    \begin{tabular}{cc}
        \faEnvelope\ \href{mailto:eason.yuchen.lee@gmail.com}{eason.yuchen.lee@gmail.com} & 
        \faPhone\ +886 903-719-328 \\[0.15cm]
        \faGithub\ \href{https://github.com/Windmill10}{github.com/Windmill10} &
        \faLinkedin\ \href{https://linkedin.com/in/yuchen-lee-47a892356}{yuchen-lee-47a892356}
    \end{tabular}
\end{center}

\vspace{0.5cm}

% About me section
\section{About Me}
Computer Science student with a strong foundation in systems development, machine learning, and hardware programming. At NTHU, I've developed skills ranging from low-level hardware programming to high-level application development. I enjoy tackling technical challenges and implementing creative solutions. Currently focused on mobile application development, machine learning, and web technologies. I am seeking an opportunity with LINE's Tech Fresh program to contribute my technical skills while gaining valuable industry experience.

% Education section
\section{Education}
\textbf{National Tsing Hua University}, Hsinchu, Taiwan \hfill \textit{Expected June 2027}\\
Bachelor of Science in Computer Science \hfill GPA: 3.89/4.30\\
\textit{Relevant Coursework:} Data Structures, Algorithms, Machine Learning, Web Development

% Technical skills section
\section{Technical Skills}
\begin{tabularx}{\textwidth}{@{}l X@{}}
\skill{Programming Languages}{C++ (Advanced), C, Python (Intermediate), Rust, Verilog, HTML/CSS (Familiar)}\\[0.2cm]
\skill{Frontend Development}{React.js, HTML5, CSS3, Streamlit}\\[0.2cm]
\skill{Machine Learning \& AI}{PyTorch, Diffusion Models, HuggingFace models, Audio processing (librosa, torchaudio)}\\[0.2cm]
\skill{Hardware Development}{FPGA Programming, SystemVerilog HDL, Xilinx Vivado, State machine architecture}\\[0.2cm]
\skill{API \& Integration}{RESTful API integration, OAuth authentication, Spotify Web API}\\[0.2cm]
\skill{DevOps \& Tools}{Git/GitHub, Anaconda, Linux, Terminal UI development}
\end{tabularx}

% Projects section
\section{Featured Projects}

\project{Bird Vocalization Generation Using Diffusion Models} \hfill \textit{Sep 2024 - Dec 2024}

\textit{A group machine learning course project that generates realistic bird vocalizations using advanced generative models.}
\begin{itemize}[leftmargin=*]
    \item Implemented data preprocessing pipeline and integrated HuggingFace sound recognition models
    \item Designed model pipeline, trained diffusion models, and handled inference/audio post-processing
    \item Led technical demonstration and delivered final project presentation to faculty panel
    \item \textbf{Technologies:} Python, PyTorch, Streamlit, librosa, torchaudio
    \item \textbf{Link:} \href{https://github.com/Windmill10/ML_diffusion}{github.com/Windmill10/ML\_diffusion}
\end{itemize}

\vspace{0.3cm}
\noindent\project{Spotify CLI} \hfill \textit{Nov 2022 - Feb 2023}
\vspace{0.3cm}

\textit{A command-line interface for Spotify built from scratch in Rust, enabling efficient daily music management without leaving the terminal.}
\begin{itemize}[leftmargin=*]
    \item Developed a terminal-based Spotify client with search, playback, and playlist management
    \item Implemented OAuth token refresh mechanism for seamless Spotify API integration
    \item Built responsive terminal UI with async operations for improved user experience
    \item \textbf{Technologies:} Rust, Spotify Web API, Terminal UI libraries
    \item \textbf{Link:} \href{https://github.com/Windmill10/Spotify_API_2}{github.com/Windmill10/Spotify\_API\_2}
\end{itemize}

\vspace{0.3cm}
\noindent\project{Slapjack Card Game on FPGA} \hfill \textit{Sep 2024 - Dec 2024}
\vspace{0.3cm}

\textit{Digital version of the Slapjack card game on dual FPGA boards, demonstrating hardware design skills and real-time systems development.}
\begin{itemize}[leftmargin=*]
    \item Designed and implemented multiplayer game on dual FPGA boards with custom protocols
    \item Created state machine architecture for game logic with different difficulty levels
    \item Developed VGA controller for graphical output and custom 8-bit music synthesizer
    \item \textbf{Technologies:} SystemVerilog HDL, Xilinx Vivado, Python
    \item \textbf{Link:} \href{https://github.com/Windmill10/HD_projects}{github.com/Windmill10/HD\_projects}
\end{itemize}

% Achievements section
\section{Achievements \& Certifications}
\begin{itemize}[leftmargin=*]
    \item \textbf{Academic Excellence:} A+ in Introduction to Programming II and Competitive Programming
    \item \textbf{Strong Performance:} A in Logic Design, Data Structures, Machine Learning
    \item \textbf{TOEIC 965} - Test of English for International Communication (2021)
\end{itemize}

% Extracurricular Activities
\section{Leadership \& Extracurricular Activities}
\begin{tabularx}{\textwidth}{@{}l X@{}}
    \textbf{HackMeiChu} & Development Team (2025) - Vue frontend development for event website \\[0.2cm]
    \textbf{DIGITIMES Hackathon} & Attendee (2025)
\end{tabularx}

% Languages section
\section{Languages}
\begin{tabularx}{\textwidth}{@{}l X@{}}
    \textbf{Mandarin Chinese} & Native \\[0.2cm]
    \textbf{English} & Professional Working Proficiency (TOEIC 965)
\end{tabularx}

% Interests section
\section{Interests}
Competitive programming, mini-app development, algorithmic problem-solving, quantitative finance, music production

\vspace{0.5cm}
\begin{center}
\textit{\color{dark}References available upon request}
\end{center}

\end{CJK}
\end{document}
